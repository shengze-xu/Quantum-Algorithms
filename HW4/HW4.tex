\documentclass[11pt]{article}
\usepackage{amssymb}
\usepackage{algorithm}  
\usepackage{algpseudocode}  
\usepackage{amsmath}  
\usepackage{graphicx}
\renewcommand{\algorithmicrequire}{\textbf{Input:}}  % Use Input in the format of Algorithm  
\renewcommand{\algorithmicensure}{\textbf{Output:}}
\parindent=22pt
\parskip=3pt
\oddsidemargin 18pt \evensidemargin 0pt
\leftmargin 1.5in
\marginparwidth 1in \marginparsep 0pt \headsep 0pt \topskip 20pt
\textheight 225mm \textwidth 148mm
\renewcommand{\baselinestretch}{1.15}
\begin{document}
\title{{\bf The Fourth Assignment}}
\author{3190102721 Xu Shengze}
\date{}
\maketitle

\begin{tabular*}{13cm}{r}
\hline
\end{tabular*}

\vskip 0.3 in

{\bf Problem 1.} Suppose that we have the following (mixed) quantum state: $$\frac{1}{3}\begin{pmatrix}
	-\frac{1}{\sqrt{2}}\\
	\frac{1}{\sqrt{2}}
\end{pmatrix}+\frac{2}{3}\begin{pmatrix}
0\\
1\\
\end{pmatrix}$$

a) Write density matrix expression for the mentioned quantum system.

b) Write expression with trace and projection and estimate the probabilities for state 0 and state 1.

\vskip 0.3 in

{\bf Answer 1.} 

a) $$
\begin{aligned}
\rho&=\frac{1}{3}\begin{pmatrix}
	-\frac{1}{\sqrt{2}}\\
	\frac{1}{\sqrt{2}}
\end{pmatrix}\begin{pmatrix}
	-\frac{1}{\sqrt{2}}&\frac{1}{\sqrt{2}}
\end{pmatrix}+\frac{2}{3}\begin{pmatrix}
	0\\
	1
\end{pmatrix}\begin{pmatrix}
	0&1
\end{pmatrix}=\begin{pmatrix}
	\frac{1}{6}&-\frac{1}{6}\\
	-\frac{1}{6}&\frac{5}{6}
\end{pmatrix}
\end{aligned}
$$

b) We have $\Pi_0=\begin{pmatrix}
	1&0\\
	0&0
\end{pmatrix}$, $\Pi_1=\begin{pmatrix}
0&0\\
0&1
\end{pmatrix}$.

The probability for state 0 is $P(|0\rangle)=\text{Tr}(\rho\Pi_0)=\text{Tr}(\begin{pmatrix}
	\frac{1}{6}&0\\
	-\frac{1}{6}&0
\end{pmatrix})=\frac{1}{6}$

The probability for state 1 is $P(|1\rangle)=\text{Tr}(\rho\Pi_1)=\text{Tr}(\begin{pmatrix}
	0&-\frac{1}{6}\\
	0&\frac{5}{6}
\end{pmatrix})=\frac{5}{6}$


\vskip 0.3 in

{\bf Problem 2.} Consider the decoherence operator D that we discussed in lecture 19. Apply D to the following quantum state represented by density matrix:
$$
\begin{pmatrix}
	\frac{3}{4}&-\frac{1}{4}\\
	-\frac{1}{4}&\frac{1}{4}\\
\end{pmatrix}
$$

Please write each step of D affecting the quantum system.

\vskip 0.3 in

{\bf Answer 2.} 

First, we append a null qubit: $\rho\to\rho\otimes |0\rangle\langle0|$, then for this quantum system, the density matrix is:
$$
\rho_0=\begin{pmatrix}
\frac{3}{4}&0&-\frac{1}{4}&0\\
0&0&0&0\\
-\frac{1}{4}&0&\frac{1}{4}&0\\
0&0&0&0
\end{pmatrix}
$$

Then we ``copy" the original qubit into the ancilla. This can be achieved by applying
the operator $\Lambda(\sigma^x):|a,b\rangle\to|a,a\otimes b\rangle$. 

We get the following formula:
$$
\rho\otimes |0\rangle\langle0|\stackrel{\Lambda(\sigma^x)}{\longrightarrow}\sum_{j,k}\rho_{j,k}|j,j\rangle\langle k,k|
$$

So for this quantum system, the density matrix becomes:
$$
\rho_0^{*}=\begin{pmatrix}
	I&0\\
	0&\sigma^x
\end{pmatrix}\rho_0\begin{pmatrix}
I&0\\
0&\sigma^x
\end{pmatrix}=\begin{pmatrix}
\frac{3}{4}&0&0&-\frac{1}{4}\\
0&0&0&0\\
0&0&0&0\\
-\frac{1}{4}&0&0&\frac{1}{4}\\
\end{pmatrix}
$$

Finally, we take the partial trace over the ancilla, which yields the diagonal matrix $\sum_{k}\rho_{kk}|k\rangle\langle k|$. 

So for this quantum system, the final diagonal matrix result is:
$$
\rho^*=(I\otimes \langle0|)\rho_0^*(I\otimes |0\rangle)+(I\otimes \langle1|)\rho_0^*(I\otimes |1\rangle)=\begin{pmatrix}
	\frac{3}{4}&0\\
	0&\frac{1}{4}
\end{pmatrix}
$$
\vskip 0.3 in

{\bf Problem 3.} Consider the topic of the lecture 21. Apply measuring operator $$\begin{pmatrix}
	1&0\\
	0&0\\
\end{pmatrix}\otimes \begin{pmatrix}
\frac{1}{\sqrt{2}}&\frac{1}{\sqrt{2}}\\
\frac{1}{\sqrt{2}}&-\frac{1}{\sqrt{2}}
\end{pmatrix}+\begin{pmatrix}
0&0\\
0&1\\
\end{pmatrix}\otimes\begin{pmatrix}
0&1\\
1&0\\
\end{pmatrix}$$
to the state $\begin{pmatrix}
	\frac{1}{2}&\frac{1}{2}\\
	\frac{1}{2}&\frac{1}{2}\\
\end{pmatrix}$. Remember that we start by joining the system with $|0\rangle\langle0|$. What is the outcome?

\vskip 0.3 in

{\bf Answer 3.} We denote $\rho=\begin{pmatrix}
	\frac{1}{2}&\frac{1}{2}\\
	\frac{1}{2}&\frac{1}{2}\\
\end{pmatrix}$. First we add subsystem: joint state is $\rho\otimes|0\rangle\langle0|$. Then we apply the measuring operator $W=\sum_j\Pi_{L_j}\otimes U_j$, we will get:
$$
\begin{aligned}
W(\rho\otimes|0\rangle\langle0|)W^{\dag}&=\sum_j(\Pi_{L_j}\rho\Pi_{L_j})\otimes(U_j|0\rangle\langle0|U_j^{\dag})\\
&=(\begin{pmatrix}
	1&0\\
	0&0\\
\end{pmatrix}\begin{pmatrix}
\frac{1}{2}&\frac{1}{2}\\
\frac{1}{2}&\frac{1}{2}\\
\end{pmatrix}\begin{pmatrix}
1&0\\
0&0\\
\end{pmatrix})\otimes(\begin{pmatrix}
\frac{1}{\sqrt{2}}&\frac{1}{\sqrt{2}}\\
\frac{1}{\sqrt{2}}&-\frac{1}{\sqrt{2}}
\end{pmatrix}|0\rangle\langle0|\begin{pmatrix}
\frac{1}{\sqrt{2}}&\frac{1}{\sqrt{2}}\\
\frac{1}{\sqrt{2}}&-\frac{1}{\sqrt{2}}
\end{pmatrix})\\
&+(\begin{pmatrix}
	0&0\\
	0&1\\
\end{pmatrix}\begin{pmatrix}
	\frac{1}{2}&\frac{1}{2}\\
	\frac{1}{2}&\frac{1}{2}\\
\end{pmatrix}\begin{pmatrix}
	0&0\\
	0&1\\
\end{pmatrix})\otimes(\begin{pmatrix}
0&1\\
1&0\\
\end{pmatrix}|0\rangle\langle0|\begin{pmatrix}
0&1\\
1&0\\
\end{pmatrix})\\
&=\begin{pmatrix}
	\frac{1}{2}&0\\
	0&0\\
\end{pmatrix}\otimes\begin{pmatrix}
\frac{1}{2}&\frac{1}{2}\\
\frac{1}{2}&\frac{1}{2}\\
\end{pmatrix}+\begin{pmatrix}
0&0\\
0&\frac{1}{2}\\
\end{pmatrix}\otimes\begin{pmatrix}
0&0\\
0&1\\
\end{pmatrix}
\end{aligned}
$$

\vskip 0.3 in

{\bf Problem 4.} Write down step-by-step application of Shor’s algorithm to the number 15. When you describe steps, use $\alpha=7$ when random number is picked in step 3.

\vskip 0.3 in

{\bf Answer 4.}

\noindent \textbf{1.} 15 is not even.

\noindent \textbf{2.} $15\neq m^k$, where $m$ is an integer and $k=2,3$, that's because $\log_2 15<4$.

\noindent \textbf{3.} Choose $a=7$, $\gcd(7, 15)=1$.

\noindent \textbf{4.} $7^4\equiv1(\text{mod}15)$, $r=\text{per}_{15}7=4$, even.

\noindent \textbf{5.} $a^{r/2}-1=7^2-1=48$, $d=\gcd(48,15)=3$, output the nontrivial divisor 3.

After the above process, the Shor’s algorithm finally outputs the nontrivial divisor 3 for the input number 15, which
means 15 is not prime.

\end{document}
